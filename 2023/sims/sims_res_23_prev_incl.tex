% Options for packages loaded elsewhere
\PassOptionsToPackage{unicode}{hyperref}
\PassOptionsToPackage{hyphens}{url}
%
\documentclass[
]{article}
\usepackage{amsmath,amssymb}
\usepackage{iftex}
\ifPDFTeX
  \usepackage[T1]{fontenc}
  \usepackage[utf8]{inputenc}
  \usepackage{textcomp} % provide euro and other symbols
\else % if luatex or xetex
  \usepackage{unicode-math} % this also loads fontspec
  \defaultfontfeatures{Scale=MatchLowercase}
  \defaultfontfeatures[\rmfamily]{Ligatures=TeX,Scale=1}
\fi
\usepackage{lmodern}
\ifPDFTeX\else
  % xetex/luatex font selection
\fi
% Use upquote if available, for straight quotes in verbatim environments
\IfFileExists{upquote.sty}{\usepackage{upquote}}{}
\IfFileExists{microtype.sty}{% use microtype if available
  \usepackage[]{microtype}
  \UseMicrotypeSet[protrusion]{basicmath} % disable protrusion for tt fonts
}{}
\makeatletter
\@ifundefined{KOMAClassName}{% if non-KOMA class
  \IfFileExists{parskip.sty}{%
    \usepackage{parskip}
  }{% else
    \setlength{\parindent}{0pt}
    \setlength{\parskip}{6pt plus 2pt minus 1pt}}
}{% if KOMA class
  \KOMAoptions{parskip=half}}
\makeatother
\usepackage{xcolor}
\usepackage[margin=1in]{geometry}
\usepackage{graphicx}
\makeatletter
\def\maxwidth{\ifdim\Gin@nat@width>\linewidth\linewidth\else\Gin@nat@width\fi}
\def\maxheight{\ifdim\Gin@nat@height>\textheight\textheight\else\Gin@nat@height\fi}
\makeatother
% Scale images if necessary, so that they will not overflow the page
% margins by default, and it is still possible to overwrite the defaults
% using explicit options in \includegraphics[width, height, ...]{}
\setkeys{Gin}{width=\maxwidth,height=\maxheight,keepaspectratio}
% Set default figure placement to htbp
\makeatletter
\def\fps@figure{htbp}
\makeatother
\setlength{\emergencystretch}{3em} % prevent overfull lines
\providecommand{\tightlist}{%
  \setlength{\itemsep}{0pt}\setlength{\parskip}{0pt}}
\setcounter{secnumdepth}{-\maxdimen} % remove section numbering
\usepackage{graphicx}
\usepackage{booktabs}
\usepackage{makecell}
\usepackage{xcolor}
\usepackage{booktabs}
\usepackage{longtable}
\usepackage{array}
\usepackage{multirow}
\usepackage{wrapfig}
\usepackage{float}
\usepackage{colortbl}
\usepackage{pdflscape}
\usepackage{tabu}
\usepackage{threeparttable}
\usepackage{threeparttablex}
\usepackage[normalem]{ulem}
\usepackage{makecell}
\usepackage{xcolor}
\ifLuaTeX
  \usepackage{selnolig}  % disable illegal ligatures
\fi
\IfFileExists{bookmark.sty}{\usepackage{bookmark}}{\usepackage{hyperref}}
\IfFileExists{xurl.sty}{\usepackage{xurl}}{} % add URL line breaks if available
\urlstyle{same}
\hypersetup{
  pdftitle={kpop Simulation Results},
  hidelinks,
  pdfcreator={LaTeX via pandoc}}

\title{kpop Simulation Results}
\author{}
\date{\vspace{-2.5em}}

\begin{document}
\maketitle

\section{Simulation Details}
\label{app:simulations}

In Section \ref{subsec:sim_intro} we briefly described our simulation
study. While \textit{kpop} achieves good balance and low bias estimates
in our application, we employ a simulation setting to more fully
investigate performance on both bias and variability. To construct a
realistic simulation, we employ the same setting above, but specify the
selection and outcome models to have direct control over the mechanism
of bias.

\hypertarget{probability-sample-selection-model}{%
\section{Probability Sample Selection
Model:}\label{probability-sample-selection-model}}

First, we specify a simplistic but non-linear selection model as
follows:

\begin{align*}
p(S=1) &= logit^{-1}\Big( PID(3way) + Age(4way)+ Gender + educ(3way) + Race(4way) \\ 
&+ BornAgain + PID(3way)*Age(4way) + BornAgain*Age(4way)\Big)
\end{align*}

Coefficients are chosen to be roughly similar to a fitted model to pew
that yields a sample size around 500. Namely:

\begin{table}[!h]

\caption{\label{tab:r5_nonlin}Non-Linear Selection Model}
\centering
\begin{tabular}[t]{lr}
\toprule
  & Coefficient Value\\
\midrule
(Intercept) & -2.00\\
pid\_3wayInd & 2.00\\
pid\_3wayRep & 2.00\\
femaleMale & 0.50\\
age\_bucket36 to 50 & 0.15\\
\addlinespace
age\_bucket51 to 64 & 0.20\\
age\_bucket65+ & 0.20\\
educ\_3wayCollege & 0.70\\
educ\_3wayPost-grad & -1.00\\
raceHispanic & 0.50\\
\addlinespace
raceOther & 0.30\\
raceWhite & 0.70\\
bornYes & 2.00\\
age\_bucket36 to 50:recode\_bornYes & 1.00\\
age\_bucket51 to 64:recode\_bornYes & 1.50\\
\addlinespace
age\_bucket65+:recode\_bornYes & 2.00\\
pid\_3wayInd:recode\_age\_bucket36 to 50 & 0.30\\
pid\_3wayRep:recode\_age\_bucket36 to 50 & 0.50\\
pid\_3wayInd:recode\_age\_bucket51 to 64 & 1.00\\
pid\_3wayRep:recode\_age\_bucket51 to 64 & 1.00\\
\addlinespace
pid\_3wayInd:recode\_age\_bucket65+ & -0.20\\
pid\_3wayRep:recode\_age\_bucket65+ & 2.00\\
\bottomrule
\end{tabular}
\end{table}

This yields the following sampling probabilities:

\begin{table}[!h]

\caption{\label{tab:r5_nonlin}Sample Inclusion Probabilities}
\centering
\begin{tabular}[t]{lr}
\toprule
  & Selection Probability\\
\midrule
Min & 0.0008\\
25\% & 0.0065\\
Mean & 0.0134\\
75\% & 0.0158\\
Max & 0.0167\\
\addlinespace
Sum & 513.1844\\
\bottomrule
\end{tabular}
\end{table}

\hypertarget{outcome-model}{%
\section{Outcome Model:}\label{outcome-model}}

To keep things straight forward, the outcome model is identical to the
selection model. In other words, again we have:

\begin{align*}
p(Vote=D) &= PID(3way) + Age(4way)+ Gender + educ(3way) + Race(4way) \\
&+ BornAgain + PID(3way)*Age(4way) + BornAgain*Age(4way)
\end{align*}

We add normally distributed noise to this outcome with mean zero and
standard deviation \(\sigma = sd(Y)*1\), yielding an \(R^2\approx.5\).
To yield negative bias, the coefficients in the outcome model start as
the inverse of the coefficients in the selection model, then through an
automated procedure they are adjusted until they produce \(\hat{y}\)'s
that lie within a probability range. This yields a population target in
percentage points of \(\bar{Y} =49.14\%\). The correlation between
selection probability and the probability of voting democratic is
\(\approx -0.6\). This induces a bias in the unweighted sample around
-3.5\%.

\begin{table}[!h]

\caption{\label{tab:r5_nonlin}Non-Linear Selection Mode with R2= 0.5 Outcome Model}
\centering
\begin{tabular}[t]{lr}
\toprule
  & Coefficient Value\\
\midrule
(Intercept) & 0.6250\\
pid\_3wayInd & -0.0750\\
pid\_3wayRep & -0.0750\\
femaleMale & -0.0188\\
age\_bucket36 to 50 & -0.0056\\
\addlinespace
age\_bucket51 to 64 & -0.0075\\
age\_bucket65+ & -0.0075\\
educ\_3wayCollege & -0.0262\\
educ\_3wayPost-grad & 0.0375\\
raceHispanic & -0.0188\\
\addlinespace
raceOther & -0.0112\\
raceWhite & -0.0262\\
bornYes & -0.0750\\
age\_bucket36 to 50:recode\_bornYes & -0.0375\\
age\_bucket51 to 64:recode\_bornYes & -0.0562\\
\addlinespace
age\_bucket65+:recode\_bornYes & -0.0750\\
pid\_3wayInd:recode\_age\_bucket36 to 50 & -0.0112\\
pid\_3wayRep:recode\_age\_bucket36 to 50 & -0.0188\\
pid\_3wayInd:recode\_age\_bucket51 to 64 & -0.0375\\
pid\_3wayRep:recode\_age\_bucket51 to 64 & -0.0375\\
\addlinespace
pid\_3wayInd:recode\_age\_bucket65+ & 0.0075\\
pid\_3wayRep:recode\_age\_bucket65+ & -0.0750\\
\bottomrule
\end{tabular}
\end{table}

\clearpage

\hypertarget{results}{%
\section{Results:}\label{results}}

We compare \textit{kpop} and \textit{kpop mf} against the same set of
mean calibration specifications discussed at length in the application
in section \ref{subsec:appmethods}. This including raking demographics,
demographics with education and raking on all variables. We also compare
performance against post stratification, stratifying on the true sample
selection model. Notably, even though our model is fairly simplistic,
the complexity of the full cross-sectional strata is such that empty
cells post a challenge for post-stratification. On average,
post-stratification must drop around 22\% of population units due to
missing strata in the sample. Finally, we include mean calibration on
the true selection model which meats the (link) linear ignorability
assumption as well as the canonical Horvitz-Thompson estimator.

\hypertarget{bias-and-mse}{%
\subsection{Bias and MSE}\label{bias-and-mse}}

The resulting bias across a full range of methods is displayed in the
table below.

\begin{table}[!h]

\caption{\label{tab:non_lin_res_race_r55_inter}Simulation Results (arranged by MSE)}
\centering
\begin{tabular}[t]{lrrr}
\toprule
  & Bias (p.p.) & MSE & Abs Bias Reduction\\
\midrule
Unweighted & -3.509 & 12.587 & 0.000\\
mean calibration (demos) & -1.607 & 2.856 & 0.542\\
mean calibration (demos+edu) & -1.291 & 1.943 & 0.632\\
mean calibration (all) & -0.029 & 0.224 & 0.992\\
kpop & -0.289 & 0.365 & 0.918\\
\addlinespace
kpop+mf (demos) & -0.162 & 0.274 & 0.954\\
kpop+mf (demos+edu) & -0.156 & 0.266 & 0.956\\
kpop+mf (all) & 0.003 & 0.239 & 0.999\\
Horvitz-Thompson (true) & -0.073 & 9.127 & 0.979\\
post-stratification (true) & -1.127 & 1.577 & 0.679\\
\addlinespace
mean calibration (true) & -0.012 & 0.211 & 0.997\\
\bottomrule
\end{tabular}
\end{table}

\hypertarget{box-plot-of-estimates-by-method}{%
\subsubsection{Box Plot of Estimates by
Method}\label{box-plot-of-estimates-by-method}}

To see these results visually, we can examine a box plot.

\includegraphics{sims_res_23_prev_incl_files/figure-latex/unnamed-chunk-6-1.pdf}

\hypertarget{standard-errors}{%
\subsection{\texorpdfstring{Standard Errors
\label{sims:SEs}}{Standard Errors }}\label{standard-errors}}

The following table present both the empirical standard errors across a
number of estimators that are reviewed and can be easily referenced in
Kott, Phillip S. ``Calibration weighting in survey sampling.'' Wiley
Interdisciplinary Reviews: Computational Statistics 8.1 (2016): 39-53.
These include including SEs that assume fixed weights, SEs that us
linearization variance estimation (eqn 15), and SEs that assume a
quasi-probability sampling process (eqn 14).

\begin{table}[!h]

\caption{\label{tab:SEs}Empirical SE Results \textbf{in Percent} for kpop Methods 1000 sims $R^2$ on Outcome =  0.497}
\centering
\begin{tabular}[t]{lrrrr}
\toprule
  & kpop & \makecell[c]{kpop+mf\\(demos)} & \makecell[r]{kpop+mf\\(d+edu)} & \makecell[l]{kpop+mf\\(all)}\\
\midrule
SE fixed & 0.532 & 0.531 & 0.527 & 0.524\\
SE linear & 0.518 & 0.515 & 0.497 & 0.476\\
SE quasi & 0.518 & 0.514 & 0.495 & 0.473\\
sd(y hat) & 0.531 & 0.498 & 0.492 & 0.490\\
\bottomrule
\end{tabular}
\end{table}

Next, we evaluate the coverage of these various SE estimators and see
all have about nominal or higher coverage.

\begin{table}[!h]

\caption{\label{tab:coverage}Bias-Adjusted SE Coverage Results for kpop Methods 1000 sims $R^2$ on Outcome =  0.497}
\centering
\begin{tabular}[t]{lrrrr}
\toprule
  & kpop & \makecell[c]{kpop+mf\\(demos)} & \makecell[r]{kpop+mf\\(d+edu)} & \makecell[l]{kpop+mf\\(all)}\\
\midrule
SE fixed & 0.957 & 0.958 & 0.959 & 0.966\\
SE linear & 0.951 & 0.950 & 0.945 & 0.949\\
SE quasi & 0.952 & 0.950 & 0.945 & 0.949\\
\bottomrule
\end{tabular}
\end{table}

\hypertarget{weights-diagnostics}{%
\subsection{Weights Diagnostics}\label{weights-diagnostics}}

The following table shows the average moments of the weights by method
across 1000 simulations. Note that \textit{"Effective SS"} refers to the
effective sample size calculated using Kish's expression
\(\frac{\left(\sum_i w_i\right)^2}{\sum_i w_i^2}\), but is not well
referenced against a set sample size since we this varied across
simulutions because we used bernoulli draws. The average sample size
should be \(\sum(p(S=1))\) (printed below).
\textit{"No. Units to 90\% Sum of Total"} is the number of units
required to sum to \(90%
\) of the total sum of the weights when weights are ordered from largest
to smallest. In other words, summing from the largest weights to the
smallest, we require this number of units to get to \(90%
\) of the total sum of the weights.

\begin{table}[!h]

\caption{\label{tab:weights}Average Moments of Weights by kpop Method across 1000 simulations}
\centering
\begin{tabular}[t]{lrrrrrr}
\toprule
Estimator & Variance & Max & Min & IQR & \makecell[r]{Effective SS\\(Kish)} & \makecell[l]{No. Units to 90\%\\Sum of Total}\\
\midrule
kpop & 0.583 & 5.969 & 0.167 & 0.692 & 329.095 & 379.443\\
kpop+mf (all) & 0.499 & 5.268 & 0.125 & 0.709 & 343.554 & 381.592\\
kpop+mf (d+edu) & 0.500 & 5.817 & 0.174 & 0.642 & 345.008 & 389.062\\
kpop+mf (demos) & 0.613 & 6.625 & 0.138 & 0.660 & 322.200 & 381.177\\
\bottomrule
\end{tabular}
\end{table}

\hypertarget{dimensions-of-k}{%
\subsection{Dimensions of K}\label{dimensions-of-k}}

\begin{table}[!h]

\caption{\label{tab:unnamed-chunk-8}Average Dimensions of K w/ R2= 0.5 Outcome Model}
\centering
\begin{tabular}[t]{lrr}
\toprule
  & Average & SD\\
\midrule
kpop & 47.260 & 23.631\\
kpop+mf (all) & 1.255 & 1.123\\
kpop+mf (demos) & 28.095 & 12.677\\
kpop+mf (demos+edu) & 15.600 & 8.427\\
\bottomrule
\end{tabular}
\end{table}

\end{document}
